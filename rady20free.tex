\documentclass[sensors,article,submit,moreauthors,pdftex]{Definitions/mdpi} 

\usepackage[utf8]{inputenc}
\usepackage{graphicx}
\usepackage{color}
\usepackage{xcolor}
\usepackage{xspace}
\usepackage{url}
\usepackage{amsmath}
\usepackage{amssymb}
\usepackage{amsthm}
\usepackage{subcaption}

\graphicspath{{figs/}}

\newcommand{\fsk}           {FSK~868~MHz}
\newcommand{\oqpsk}         {O-QPSK~2.4~GHz}
\newcommand{\ofdm}          {OFDM~868~MHz}
\newcommand{\openradio}     {\texttt{openradio}}
\newcommand{\figwidth}      {0.80}

\newcommand{\etal}          {\textit{et al.}}

\newcommand{\todo}[1]       {\textbf{\textcolor{red}{[TODO] #1}}}
\newcommand{\lorem}         {\textcolor{green}{Lorem ipsum dolor sit amet, consectetur adipiscing elit, sed do eiusmod tempor incididunt ut labore et dolore magna aliqua. Ut enim ad minim veniam, quis nostrud exercitation ullamco laboris nisi ut aliquip ex ea commodo consequat. Duis aute irure dolor in reprehenderit in voluptate velit esse cillum dolore eu fugiat nulla pariatur. Excepteur sint occaecat cupidatat non proident, sunt in culpa qui officia deserunt mollit anim id est laborum.}}
\newcommand{\mina}[1]       {\textbf{\textcolor{blue}{[Mina] #1}}}
\newcommand{\quentin}[1]    {\textbf{\textcolor{blue}{[Quentin] #1}}}
\newcommand{\dominique}[1]  {\textbf{\textcolor{blue}{[Dominique] #1}}}
\newcommand{\thomas}[1]     {\textbf{\textcolor{blue}{[Thomas] #1}}}

\Title{No Free Lunch: Characterizing the Performance of 6TiSCH when using Different Physical Layers}

\newcommand{\orcidauthorA}{0000-0003-2529-8272}
\newcommand{\orcidauthorB}{0000-0002-3695-9315}

\Author{
    Mina~Rady$^{1,2}$\orcidA{},
    Quentin~Lampin$^{1}$,
    Dominique~Barthel$^{1}$,
    Thomas~Watteyne$^{2}$\orcidB{}
}

\AuthorNames{Mina~Rady, Quentin~Lampin, Dominique~Barthel, Thomas~Watteyne}

\address{%
    $^{1}$ \quad Orange Labs, Meylan, France; \{mina1.rady,quentin.lampin,dominique.barthel\}@orange.com\\
    $^{2}$ \quad Inria, EVA Team, Paris, France; thomas.watteyne@inria.fr
}

\corres{Correspondence: mina1.rady@orange.com}

\abstract{
    Low-power wireless applications require different trade-off points between latency, reliability, data rate and power consumption.
    Given such a set of constraints which physical layer should I be using?
    We study this question in the context of 6TiSCH, a state-of-the-art recently standardized protocol stack developed for harsh industrial applications.
    Specifically, we augment OpenWSN, the reference 6TiSCH open-source implementation, to support one of three physical layers from the IEEE802.15.4g standard:
        \fsk\   which offers long range,
        \ofdm\   which offers high data rate, and
        \oqpsk\ which offers more balanced performance.
    We run the result firmware on the 42-mote OpenTestbed deployed in an office environment, once for each physical layer.
    Performance results show that, indeed, no physical outperforms the other all metrics.
    This article argues for combining the physical layers, rather than choosing one, in a generalized 6TiSCH architecture in which technology-agile radio chips (or which there are now many) are driven by a protocol stack which chooses the most appropriate physical layer on a frame-by-frame basis.
}

\keyword{
    6TiSCH,
    Performance Evaluation,
    OpenWSN,
    Testbed,
    Technology Agility.
}

\begin{document}

%==============================================================================
\section{Introduction}
\label{sec:introduction}

% low-power wireless use cases

Applications for low-power wireless technology are numerous, including
    smart building~\cite{munoz18overview,kazmi14review},
    environmental monitoring~\cite{munoz18evaluation},
    precision agriculture~\cite{watteyne16peach},
    automated meter reading~\cite{sum17experimental},
    indoor localization~\cite{tanaka20blink},
    smart grid~\cite{fadel15survey} and
    predictive maintenance~\cite{civerchia17industrial}.

% PHY layers

Several physical layers (PHYs) exist, each finding a different trade off between
    energy consumption,
    datarate and
    range.
One particularly interesting development in low-power wireless is the standardization in 2016 of the IEEE802.15.4g amendment~\cite{std_ieee802154g}, which defines additional PHY layer for the older IEEE802.15.4 standard~\cite{std_ieee802154}.
IEEE802.15.4g defines PHY specifications initially for smart metering utility networks.
IEEE802.15.4g defines
    three families of modulations (Frequency Shift Keying -- FSK,  Offset Quadrature Phase-Shift Keying -- O-QPSK, Orthogonal Frequency Division Multiplexing -- OFDM),
    data rate ranging from 25~kbps to 800~kbps, and
    two frequency bands (sub-GHz and 2.4~GHz).
These options result in a large set of combinations to address a wide variety of applications, from
    small home installations~\cite{tuset19experimental} to
    kilometer-scale smart agriculture deployments~\cite{munoz19km}.
In the remainder of this article, we call a ``PHY'' a radio configuration, consisting of
    a modulation,
    a data rate and
    a frequency band.

% typical approach

A traditional approach to having this array of choices is to
    characterize each PHY,
    select the one that best corresponds to the application, and
    build a network solution using that PHY.
Two elements allow for a more dynamic approach:
    technology-agile radio chips and
    TSCH.

% technology-agile radio chips

As the processes used in Integrated Circuits advance, chip manufacturers add more features into their products for the same die size,
    maintaining low-power operation and
    competitive pricing.
System-on-Chip (SoC), such as Texas Instruments' CC2538~\cite{datasheet_cc2538}, which combine a micro-controller and a radio in the same die are now commonplace.
Advanced SoC such as the Texas Instruments' CC2650~\cite{datasheet_cc2650} contain a co-processor which, combined with a 2.4~GHz front-end, implements either the Bluetooth Low Energy (BLE) standards.
And while the CC2650 requires that co-processor to be reprogrammed to switch from one to another,
    the newer nRF52840~\cite{datasheet_nrf52840} from Nordic Semiconductor allow switching between these standards on a frame-by-frame basis.
The AT86RF215 from Atmel arguably goes one step further in that it implements the entire set of modulations, data rates and frequency band defined in the IEEE802.15.4 standard.
Using such technology-agile radio chips opens up many possibilities to dynamically match the PHY used to the application.

% TSCH

While early low-power wireless networks used contention-based approach,
    Time Synchronized Channel Hopping (TSCH), initially developed for industrial applications,
    is now supported by many commercial products and open-source implementations~\cite{watteyne16industrial}.
In a TSCH network, time is cut into timeslots and a schedule orchestrates all communication;
    it indicates to every node what to in each timeslot: transmit, listen or sleep.
TSCH is the base principle of standards such as WirelessHART, ISA100.11a, IEEE802.15.4, and, more recently, 6TiSCH, a set of IETF standard which combines the industrial performance of TSCH with the ease of use of IPv6~\cite{draft-ietf-6tisch-msf}.
In today's TSCH networks, for each cell, the schedule indicates the frequency to communicate on, resulting in ``channel hopping'', a technique known to efficiently combat external interference and multi-path fading.

% vision

Our long-term vision is that low-power wireless network will combine the technology-agility of modern radio chips which TSCH-like scheduling approaches so nodes dynamically change their PHY layer on a frame-by-frame basis.
That is, if the next hop is far,   use a long-range (but slower and more energy-hungry) PHY.
         If the next hop is close, use a fast PHY so the radio can be switched of as fast as possible.

% goal

The goal of this article is to get one step closer to this vision by comparing the performance of a 6TiSCH network for different PHY layers.
This is a necessary first step towards dynamically changing the PHY layer,
    the object of our future research.
Specifically, we augment the OpenWSN reference 6TiSCH implementation to support the following three PHYs:
    \oqpsk\ at 250~kbps,
    \fsk\ option 1 at 50~kbps,
    \ofdm\  option 1 MCS 3 at 800~kbps.
The literature indicates that these three PHYs cover the range of possibilities of IEEE802.15.4g:
    very high data rate with \ofdm,
    very long range with \fsk,
    \oqpsk\ being a balance between range and data rate~\cite{munoz19km,draft-munoz-6tisch-multi-phy-nodes,brachmann19ieee}.
We then conduct a comprehensive experimental campaign on the OpenTestbed, a 42-node testbed of OpenMote~B boards~\cite{tuset16openmote}, a platform which features the AT86RF215 and CC2538 chips.
We compare the performance of the network in terms of
    network formation time,
    battery lifetime,
    end-to-end latency and
    end-to-end reliability.

% remainder

The remainder of this article is organized as follows.
Section~\ref{sec:related} surveys the most relevant related work.
Section~\ref{sec:contributions} clearly states the problems and lists the contributions of this article.
Section~\ref{sec:openwsn} introduces the agile extension to the OpenWSN PHY-layer that enabled this research.
Section~\ref{sec:opentested} introduces the testbed used for the experimental campaigns of this paper.
Section~\ref{sec:methodology} introduces the methodology of the experiments including the used metrics and KPIs.
Section~\ref{sec:results} demonstrates the experiment results and KPI evaluations.
Finally, section~\ref{sec:conclusion} provides insights and conclusions based on the results.

%==============================================================================
\section{Related Work}
\label{sec:related}

% organization

This section surveys related work on
    performance improvement and evaluation of the 6TiSCH protocol stack (Section~\ref{sec:related_6tisch}),
    performance evaluation of IEEE802.15.4g family of modulations (Section~\ref{sec:related_4g}), and
    hybrid radio utilization in the 6TiSCH stack  (Section~\ref{sec:related_hybrid}). 

%------------------------------------------------------------------------------
\subsection{Performance Improvement and Evaluation of 6TiSCH}
\label{sec:related_6tisch}

Yang~\etal~\cite{yang18analysis} evaluate the performance of the full 6TiSCH stack on the OpenWSN reference architecture in terms of responsiveness to time critical event triggers.
They vary the number of active slots in a frame of length 11 slots and measure how packet end-to-end latency is affected.

Theoleyre~\etal~\cite{theoleyre16experimental} evaluate the performance of the 6TiSCH stack with the deployment of traffic isolation mechanisms that allow reservation of dedicated slots for certain applications and reservation of shared slots for alarm events.
They report end-to-end latency and PDR under various schedule management strategies,
    including using uniformly distributed shared cells instead of contiguous (adjacent) shared cells in the slot-frame.

Teles Hermeto~\etal~\cite{teleshermeto18reactions} execute a performance evaluation of the 6TiSCH protocol stack in an indoor environment with a focus on the stability of the stack performance.
They report the end-to-end reliability and the number of parent changes as the network is converging.
They also propose a stable link quality metric and a simplified method for schedule inconsistency management. 

Ben Yaala~\etal~\cite{benyaala16performance} evaluate the performance of the 6TiSCH stack in co-located networks.
They consider both cases where the coexisting 6TiSCH networks are either synchronized or un-synchronized.

In all of these related work, the PHY used is IEEE802.15.4 \oqpsk.
Sum~\etal~\cite{sum17experimental} are the exception, as they provide an experimental evaluation of IEEE802.15.4e Time Slotted Channel Hopping MAC based on IEEE802.15.4g \fsk\ radio.
The experiments report range and performance testing results in terms of PDR and PER in four situations:
    LOS conditions,
    NLOS conditions,
    tree topology in corridor setting and
    line topology across buildings.

%------------------------------------------------------------------------------
\subsection{IEEE802.15.4g Performance Evaluation}
\label{sec:related_4g}

Some related work evaluate the performance of IEEE802.15.4g PHYs.

Kojima~\etal~\cite{kojima15system} examine the impact of interference between MR-FSK mode 2 and MR-OFDM option 4 MCS 3. 
Authors deploy IEEE802.15.4e MAC with multihop capability on top of each PHY layer and measure the impact of the interference between two networks, each running one PHY.
Results are reported in terms of the degradation of throughput of each network in the presence of the other interfering network.
\todo{}

Mu\~noz~\etal~\cite{munoz18overview} run an experimental campaign to compare the performance of
    IEEE802.15.4 \oqpsk\ at 250~kbps, and
    IEEE802.15.4g OFDM.
They show a higher robustness of OFDM, even though it operates as a higher bit rate (800~kbps).
They show that, although a radio draws less current when running \oqpsk\, using \ofdm\ leads to an overall lower power budget as transmission happens much faster.

The same authors also evaluate the performance of all IEEE802.15.4g PHYs~\cite{munoz18evaluation}.
They conduct an complete range-testing campaigns for the 31 PHYs on the four scenarios they consider the most prevalent in outdoor applications:
    line of sight,
    smart agriculture,
    urban canyon,
    smart metering. 
Results of the range-tests are reported in terms of PDR measurements, throughput, and electric charge consumption.
They demonstrate the longer range of FSK and O-QPSK in the sub-GHz band compared to OFDM options due to their higher receiver sensitivity.  


These works do not address the performance of a full 6TiSCH stack using these PHY layers.  

%------------------------------------------------------------------------------
\subsection{Hybrid Radio Utilization}
\label{sec:related_hybrid}

Brachmann~\etal~\cite{brachmann19ieee} integrate multiple PHY layers within the 6TiSCH stack.
The authors perform range-testing measurements on multiple modulations:
     2-GFSK (at 1.2, 8, 50, and 250~kbps),
     4-GFSK (at 1,000~kbps),
     \oqpsk\ (at   250~kbps). 
They report the link quality in terms of number of motes reached by each mote in the network, and link symmetry.
The authors then choose two modulations in the sub-GHz band for integration in the 6TiSCH stack:
    2-GFSK at 1.2~kbps for transmitting Enhanced Beacons (EB),
    4-GFSK at 1,000~kbps for data traffic. 
This allows for EB sent by the gateway to reach more nodes, resulting in faster and more accurate network-wide synchronization compared to multi-hop transmissions. 
The slot templates for the PHY layers are designed in accordance with the IEEE802.15.4 standard,
    and ported on Contiki-NG implementation of 6TiSCH. 
The authors report on the performance of the multi-PHY network in terms of channel utilization and the distribution of synchronization accuracy in the network, in comparison with the distribution of hop-count. 
They demonstrate that a faster bit-rate of 4-GFSK at 1,000~kbps for data packets diminishes the probability for collision and leads to less than 0.1\% channel utilization despite the increased re-transmission rate.
At the same time, the slower bit-rate of 2-GFSK at 1.2~kbps for EB packets leads to 20$\times$ higher channel occupancy,
    at 2\% duty cycle (leading to violation of the license-free band regulation in many regions). 

This research paves the road for a thorough investigation of multi-PHY integration in the 6TiSCH stack.
However, it explicitly assumes an inverse correlation between bit-rate and range.
While this could be true for the specific subset of modulations selected for evaluation, it is not in concordance with research results in~\cite{munoz18overview} and~\cite{munoz18evaluation}, which show that a high bit-rate modulation such as OFDM 1 MCS 3 at 800~kbps can offer competitive performance compared to 2-FSK at 50~kbps in terms of range, reliability, and duty cycle. 
This is a significant observation because if long-range radio is needed for EB synchronization as~\cite{brachmann19ieee} suggests, higher bit-rates are appreciated for EBs in order to minimize energy consumption and also collision probability since EBs are transmitted on shared cells. 
Furthermore, it still remains unclear whether there are any side-effects to the choice of one radio or another  on the end-to-end performance as a whole so that the costs and benefits of each PHY layer choice can be fully assessed.

%==============================================================================
\section{Problem Statement and Contributions}
\label{sec:contributions}

% problem statement

While active research proposed various enhancements in 6TiSCH  for achieving wire-like reliability and low power consumption,
    one common assumption is that it uses the IEEE802.15.4 O-QPSK at 2.4~GHz.
While \oqpsk\ is appropriate for many applications,
   the increasing demands for long range (including for environmental monitoring and automated meter reading) has triggered the standardization of longer range PHYs~\cite{munoz18evaluation}.
A problem statement has been proposed in the Internet Engineering Task Force (IETF) detailing the theoretical challenges or side-effects of augmenting 6TiSCH to include different PHYs~\cite{draft-munoz-6tisch-multi-phy-nodes}. 
Using a different PHY will cause the network to behave differently, leading to different overall performance.
For example, a different energy consumption and link cost changes how links to neighbors are evaluated at the link-layer,
    hence which multi-hop paths are picked by the routing protocol.
A different PHY also impacts
    the number of re-transmissions,
    the level of interference and
    the amount of contention is shared cells~\cite{kim15hybrid}.
Changing the PHY layer triggers subtle changes in the behavior of the protocol stack,
    resulting in different performance.
We believe measuring this different is best done through a real-world system level evaluation~\cite{saltzer84endtoend}.
We show that there are significant benefits of each of the radio options: \fsk, \ofdm, and \oqpsk, yet they all come at a certain cost.

% contributions

The contribution of this article is three-fold:

\begin{itemize}
    \item We augment the OpenWSN reference implementation of 6TiSCH to support the three PHY layers\footnote{~As an online addition to this paper, these extensions have been merged into the main codebase of OpenWSN at \url{https://github.com/openwsn-berkeley} and published under an open-source license.}.
    \item We conduct an experimental performance evaluation campaign of the resulting code on the 42-note OpenTestbed.
        For each PHY, we measure 
            the network formation time,
            the end-to-end reliability,
            the end-to-end latency, 
            the estimated battery lifetime of each node, and
            the memory footprint of the implementation.
    \item We highlight the advantages and disadvantages of each PHY layer, specifically:
        \begin{itemize}
            \item Using \fsk\ yields
                the fastest network formation,
                the highest end-to-end reliability, and
                the lowest end-to-end latency,
                    at the cost of
                        a 10$\times$ decreased battery lifetime compared to \oqpsk.
            \item Using \oqpsk\ yields
                the longest battery lifetime,
                    at the cost of
                        the slowest network formation,
                        the lowest end-to-end reliability, and
                        the highest end-to-end latency.
            \item Using \ofdm\ yields
                an intermediate network formation time,
                end-to-end reliability, and
                end-to-end latency
                    at the cost of
                        a 5$\times$ decreased battery lifetime compared to \oqpsk.
                In line with~\cite{munoz18evaluation}, a node using \ofdm\ discovers number of neighbors closer to \fsk\ than \oqpsk\,
                    despite being much higher data rate.
        \end{itemize}
        This shows there is no PHY that is best across all metrics.
\end{itemize}

%==============================================================================
\section{A PHY-layer Agile Extension of OpenWSN}
\label{sec:openwsn}

% intro to 6TiSCH

The IETF ``IPv6 over the TSCH mode of IEEE802.15.4e'' 6TiSCH working group has standardized how to combine the TSCH and IPv6~\cite{vilajosana19ietf}.
The 6TiSCH protocol stack is rooted in IEEE802.15.4 at the PHY, and the TSCH mode of IEEE802.15.4e at the link layer.
Multi-hop capability is supported at the routing layer via the IPv6 routing protocol for Lower-Power and Lossy Networks (RPL).
Furthermore, 6LoWPAN is an adaptation layer for enabling the interfacing between the routing layer and the specific operations of the MAC layer such as neighbor discovery, registration, ranking, and maintenance operations that impact how routes are calculated. 
The 6TiSCH protocol stack has been ported to all main open-source IoT protocol stacks: OpenWSN, Contiki-NG, RIOT and TinyOS~\cite{watteyne16industrial}. 

% intro to OpenWSN

Since it is the reference implementation, and since it is ported to the OpenMote~B, we use the OpenWSN implementation in this article.
Prior to this work, OpenWSN only supported the IEEE802.15.4 O-QPSK 2.4~GHz PHY.
Initial range measurements were using the OpenMote's AT86RF215 radio~\cite{munoz18evaluation,munoz18overview}, but the full 6TiSCH stack was never ported.
We therefore extend OpenWSN to support the following PHYs: \fsk, \ofdm, \oqpsk.
This extension consist of three steps:
     writing the low-level drivers to configure the radio chips appropriately,
     tuning the durations within the timeslots to each PHY, and
     providing a clean software abstraction to allow the TSCH implementation to switch between them.

% PHY layer

Table~\ref{tab:phys} lists the main characteristics of the PHY layer tested;
    TX power and sensitivity number are taken for the chip used.

\begin{table}
    \centering
    \begin{tabular}{|r|r|r|r|r|r|}
        \hline
                & Radio chip & Data rate  & TX power        & Sensitivity   &   Link budget \\ \hline
        \fsk    &  AT86RF215 &   50~kbps  &  +14.5~dBm      &    -114~dBm   &   128.5~dB \\ \hline
        \ofdm   &  AT86RF215 &   800~kbps &  +10.0~dBm      &    -104~dBm   &   114.0~dB \\ \hline
        \oqpsk  &  CC2538    &   250~kbps &  +07.0~dBm      &     -97~dBm   &   104.0~dB \\ \hline
    \end{tabular}
    \caption{The PHY layers tested.}
    \label{tab:phys}
\end{table}

% OpenWSN Extension

We extend OpenWSN with a generic \openradio\ interface.
This interface driver implements the same set of functions called by the TSCH state machine, and maps those to radio driver corresponding to the PHY being used.
We further introduce a timeslot template for each PHY.
The ``timeslot template'' consists of the appropriate timeslot timings, which we fine tuned using a logic analyzer.
The duration of SPI transactions, the datarate of the communication or the turn-around time of the radio all influence in the timeslot template.
Fig.~\ref{fig:timeslot_templates} illustrates the different timeslot templates when transmitting a 100~B data payload with acknowledgement.
Red markers highlight the approximate time for SPI communications (non in CC2538).
A 40~ms slot template duration is used for options, allow for fair comparison between all three PHYs.

\begin{figure}
	\centering
	\includegraphics[width=\figwidth\columnwidth]{timeslot_templates}
	\caption{
	    Timeslot templates for the three~PHYs.
	    We use a 40~ms timeslot duration in all cases.
	}
    \label{fig:timeslot_templates}
\end{figure}

% parameters

We use a slotframe length of~41 timeslots.
Table~\ref{tab:stack_params} summarizes the parameters of the stack.

\begin{table}
\centering
    \begin{tabular}{|l|r|}
        \hline
        Parameter                           &   Value \\ \hline
        Application traffic period          &    60~s \\
        RPL DIO period                      &    10~s \\
        RPL DAO period                      &    60~s \\
        neighbor table size                 &      45 \\
        packet queue size                   &      15 \\
        slotframe length                    &      41 \\
        timeslot duration                   &   40~ms \\
        EB probability                      &    10\% \\
        Number of radio channels            &      16 \\
        New neighbor RSSI threshold         & -80~dBm \\
        Max num. re-transmissions           &      15 \\ \hline
    \end{tabular}
    \caption{Parameters of the OpenWSN protocol stack.}
    \label{tab:stack_params}
\end{table}

% result: footprint

Before this work, the memory footprint of the entire OpenWSN stack was 42~kB.
This increases to 77~kB with the additions listed above, in particular with both the CC2538 and AT86RF215 drivers.
This is a very small footprint which comfortably fits in modern SoC which all have 512~kB of flash, or more.

%==============================================================================
\section{The OpenTestbed}
\label{sec:opentested}

% intro testbed, architecture

We use the OpenTestbed~\cite{munoz19opentestbed} to extract the performance of the OpenWSN stack with the extensions detailed in Section~\ref{sec:openwsn}.
The OpenTestbed is composed of 42~OpenMote~B boards deployed across an office floor at the Inria research center in Paris in groups of 4 (see floorplan on Fig.~\ref{fig:building_motes}).
The distribution of the motes happens in clusters of 4-18 motes, mimicking nodes clustered around machines in an industrial setting.
The OpenTestbed is built from off-the-shelf components,
    networked together using the building's 5~GHz Wi-Fi network (i.e. no dedicated Ethernet network),
    without requiring back-end servers, and
    with an open access interface.
Interaction with the OpenTestbed is done entirely over MQTT.
The Raspberry Pi single-board computers to which the OpenMote~B boards are attached implement a number of commands to reprogram and reset the boards.
During an experiment, one can interact (read/write) with the serial port of each OpenMote~B board, in real-time, using MQTT messages.

\begin{figure}
	\centering
	\includegraphics[width=\figwidth\columnwidth]{building_motes}
	\caption{Location of the 42~motes of the OpenTestbed across an office floor at Inria-Paris.}
    \label{fig:building_motes}
\end{figure}

% OpenMote

As show in Fig.~\ref{fig:mote_ot}, the OpenMote~B is an IoT platform which features both
    a CC2538 (a micro-controller and \oqpsk\ radio) and
    an AT86RF215 (an \fsk\ and \ofdm\  radio)~\cite{tuset16openmote}.
There are two antennas:
    a 2.4~GHz antenna connected to the CC2538,
    a sub-GHz antenna connected to the AT86RF215.
The ARM Cortex-M3 on the CC2538 features 32~bB of RAM and 512~kB of flash.
The OpenWSN firmware is loaded onto that micro-controller;
    that firmware then interacts with the CC2538's radio directly using the registers,
    and with the AT86RF215 over SPI.
We use a combination of JTAG in-circuit debugging and a logic analyzer to debug the code.

\begin{figure}
	\centering
	\includegraphics[width=0.90\columnwidth]{mote_ot}
	\caption{
	    The OpenMote~B \textit{(left)}; the OpenTestbox, part of the OpenTestbed \textit{(right)}.
	}
    \label{fig:mote_ot}
\end{figure}

%==============================================================================
\section{Methodology}
\label{sec:methodology}

% one experiment

We run the OpenWSN network once for each of the three PHYs.
During the experiments, the floor is mostly unoccupied, so we don't expect WiFi interference beyond the regular 100~ms beaconing interval of the 8~WiFi access points on that floor~\cite{munoz18overview}.
Each time, we load the firmware onto the testbed, the switch on the OpenVisualizer software, which connects to all motes over their serial port (over MQTT).
In the OpenVisualizer, we the select the mote which we want to play the role of DAG root.
We always select the same mote, shown in Fig.~\ref{fig:building_motes}, which is positioned at the center of the floor in room~A.
We then let the network run for 90~min, recording all of data generated by the motes.

% application layer

We developed a custom-built application which runs at the application layer of each mote.
That application sends a data packet every minute containing the following fields:

\begin{itemize}
    \item a counter we use to detect packets lost,
    \item the time at which the packet is generated (Asynchronous Slot Number ASN),
    \item the DAG rank of the sender,
    \item the size of the neighbor of the sender,
    \item $T_{on}$ how long the sender's radio has been on receiving since the previous transmission,
    \item $T_{TX}$ how long the sender's radio has been on transmitting since the previous transmission,
    \item $T_{total}$ the amount of time sine the previous transmission,
    \item the maximum and minimum number of packets in the packet buffer since the previous packet.
\end{itemize}

% KPIs

After the experiment is done, we use that information to compute the following Key Performance Indicators (these KPIs are recommended by~\cite{vucinic20key}):

\begin{itemize}
    \item Network Formation Time:
        how much time between the moment the root is selected and that root has received at least one packet from \textit{each} mote. 
    \item End-to-End Reliability:
        what portion of all packet generated in the network are received by the root. 
    \item End-to-End latency:
        how much time between the generation of the packet at the sender, and reception at the root.
    \item Radio duty cycle:
        what portion of the time is a node's radio on; a good indication for its power consumption.
\end{itemize}

% running the experiments

We display the results in three forms:

\begin{itemize}
    \item Time series.
        We show the mean and the inter-quartile range of the KPI,
            with a 3~min sliding window and a 1~s time resolution. 
    \item Cumulative Distribution Function (CDF).
        We show the cumulative distribution of all data samples at steady state (starting 30~min into the experiment), using 100~bins.
    \item Probability Distribution Function (PDF).
        We show the probability distribution of all data samples at steady state using 100~bins.
        This allows for a more detailed assessment of the distribution of each data set. 
\end{itemize}

%==============================================================================
\section{Experimental Results}
\label{sec:results}

This section summarizes our key findings, focusing on
    network formation time (Section~\ref{sec:res_formation}),
    end-to-end reliability (Section~\ref{sec:res_reliability}),
    end-to-end latency (Section~\ref{sec:res_latency}), and
    battery lifetime (Section~\ref{sec:res_lifetime}).

%------------------------------------------------------------------------------
\subsection{Network Formation Time}
\label{sec:res_formation}

% definition

The network formation time is measured between the moment the DAG root is selected, and the moment the DAG root has received a data packet from each mote.
The network formation process encompasses
    the time it takes for a motes to synchronize to the network,
    the time it takes for it to complete a security handshake and
    the time it takes for it to acquire a rank.
It is the time a user would have to wait for their network to be fully functional.

% worst case

This is a worst case setup, as we turn on all the motes first, and the gateway last.
Per the 6TiSCH standard, traffic generated by the motes during their secure handshake uses shared cells.
All motes trying to join approximately at the same time will cause a lot of contention of these shared cells, increasing the network formation time.

% network formation time

Fig.~\ref{fig:time_firstpacket_cdf} shows that the \fsk\, \ofdm\  , and \oqpsk\ network is 90\%-formed in 7, 9, and 11~min, respectively.
The higher the link budget (i.e., the longer the range), the faster the network forms.

\begin{figure}
	\centering
	\includegraphics[width=\figwidth\columnwidth]{time_firstpacket_cdf}
	\caption{The network tends to form faster when using a longer-range PHY.}
    \label{fig:time_firstpacket_cdf}
\end{figure}

% num. neighbors over time

To understand the impact of PHY on network formation time, we plotted in Fig.~\ref{fig:neighbors_time} the number of neighbors evolving over time.
Because of its long range, a mote using \fsk\ tends to discover more nodes, faster than a node using \oqpsk.
This observation exactly follows the link budget from Table~\ref{tab:phys}.
Discovering many neighbors is helpful in two ways.
First, it allows a mote to quickly hear a node that is already part of the network,
    hence to synchronize quickly.
Second, it gives a mote a higher probability of joining through a neighbor closer to the root; this decreases the number of hops necessary for joining.

\begin{figure}
	\centering
	\includegraphics[width=0.85\columnwidth]{neighbors_time}
    \caption{Motes discover more neighbors faster when using a longer-range PHY.}
    \label{fig:neighbors_time}
\end{figure}

% neighbor table overflow

However, having a longer range increases the risk of neighbor table overflow.
For example, if the neighbor table can hold up to 10~entries, what is the appropriate behavior when a mote  hears 15~other motes?
The challenge is that, without having another mote in its neighbor table, a mote cannot keep statistics to elect the ``best'' neighbors, and has to to make decisions on partial data, such as a single RSSI value.
We note that none of the standards makes clear recommendation for neighbor table grooming.

% rank over time

Fig.~\ref{fig:dagrank_time} gives some insights into the network formation itself.
It plots how the rank reported by the motes evolves over time.
Per the RPL and MSF standards~\cite{rfc6550,draft-ietf-6tisch-msf}, a mote's rank is computed using~\eqref{eq:rank}, where $numTx$ and $numAck$ are counters of the number of transmission attempts and transmission successes to a neighbor, respectively.
$minHopIncrease = 256$ is the rank increase if the link to the mote's routing parent is ideal.
If that link becomes lossy (i.e.~$numTx/numAck>1$), the ``cost'' associated with the link increases, and the mote's rank increases accordingly.

\begin{figure}
	\centering
	\includegraphics[width=0.85\columnwidth]{dagrank_time}
    \caption{Contention and slow neighbor discovery cause the nodes' rank to be artificially high at the beginning of network formation.}
    \label{fig:dagrank_time}
\end{figure}

\begin{equation}
    rank = ((3 \cdot \frac{numTx}{numAck})-2)-minHopIncrease
    \label{eq:rank}
\end{equation}

The first portion of Fig.~\ref{fig:dagrank_time} ($t<50s$) show that the nodes start by having a very high rank.
This is caused by two phenomena, combined.
First, a mote may not discover its neighbor with the lowest rank immediately, and instead join with a artificially high rank; this resolves over time as the network stabilizes by becoming shallower.
Second, all motes attempting to join create contention on the minimal cell, causing
    transmissions to fail,
    $numTx/numAck$ to increase, and
    the motes' rank to increase.
This, again, resolves as the network stabilizes by having less contention on the shared cells.
These two phenomena compete: the longer the PHY range, the shallower the initial network, but the higher the contention.
Fig.~\ref{fig:dagrank_time} shows that, once the network has stabilized, it tends to be shallower (smaller DAG rank) when using a longer-range PHY.

%------------------------------------------------------------------------------
\subsection{End-to-End Reliability}
\label{sec:res_reliability}

% definition

We call end-to-end reliability the portion of UDP datagrams of each mote that reach the root, and use the counter in the datagrams to compute it.
Table~\ref{tab:pdr_table} shows PDR statistics over the last 15~min of the experiments, computed for all motes in the network.
We expect close to 100\% PDR in all cases (commercial TSCH implementation offer >99.999\% PDR~\cite{vucinic20key}).

\begin{table}
    \centering
    \begin{tabular}{|r|r|r|r|r|r|r|}
        \hline
                &      Min &  Average &   Median &      Max &  StDev & Active motes \\ \hline
        \fsk    &  100.0\% &  100.0\% &  100.0\% &  100.0\% &  0.0\% &       92.8\% \\ \hline
        \ofdm   &   93.8\% &   99.8\% &  100.0\% &  100.0\% &  0.9\% &      100.0\% \\ \hline
        \oqpsk  &   73.3\% &   98.1\% &  100.0\% &  100.0\% &  6.0\% &       97.6\% \\ \hline
    \end{tabular}
    \caption{End-to-end PDR statistics over all motes in the network, computed over the last 15~min of the experiments.}
    \label{tab:pdr_table}
\end{table}

% queue occupancy

Besides possible bugs in our implementation, packet drop might be happening because of queue overflow.
As shown in Section~\ref{sec:res_formation}, a lower link budget leads to higher DAG rank, and more relaying, which increases the likelihood of filling up the packet buffer and dropping packets.
Fig.~\ref{fig:maxBufferSize_cdf_plot_full_steady_15} show the Cumulative Density Function of the queue occupancy values reported by the motes, in the same period corresponding to Table~\ref{tab:pdr_table}.
Even though Table~\ref{tab:stack_params} correctly indicates the buffer can hold up to 15~packets, 5~additional buffer entries are reserved for control frames such as Enhanced Beacons and acknowledgements.
When the number of occupied entries in the buffer exceeds 15, more data packets cannot be admitted into the buffer and are dropped .
Motes running \oqpsk\ and \ofdm\ drop packets 3\% and 1\% of the time, respectively.
\fsk\ does not suffer buffer overflow, as the network is much shallower.

\begin{figure}
	\centering
	\includegraphics[width=\figwidth\columnwidth]{maxBufferSize_cdf_plot_full_steady_15}
	\caption{CDF of buffer occupancy over the last 15~mins of the experiment. Having more than 15~entries occupied in the buffer (the red line) leads to data packet drops.}
    \label{fig:maxBufferSize_cdf_plot_full_steady_15}
\end{figure}

%------------------------------------------------------------------------------
\subsection{End-to-End Latency}
\label{sec:res_latency}

% definition

We call end-to-end latency the amount of time between the moment a mote generates a new UDP datagram, and the moment it reaches the root.
It is computed at the root using a timestamp inside the datagram.

% latency over time

Fig.~\ref{fig:latency_time} plots the evolution of latency over time.
As shown in Fig.~\ref{fig:latency_cdf}, 90\% of the data reaches the root after 10, 25, and 35~s for \fsk, \ofdm\ and \oqpsk, respectively.
Fig.~\ref{fig:latency_pdf} shows a long-tailed distribution of latency for shorter range PHYs.

% discussion

With 41 slots per slotframe and 40~ms timeslot, each mote on average gets one transmission opportunity every 1.68~s.
Given the increased number of hops for lower range PHYs, the higher latency of \oqpsk, compared to that of \fsk, was to be expected.

\begin{figure}
	\centering
	\begin{subfigure}{0.49\columnwidth}
    	\centering
    	\includegraphics[width=1.00\columnwidth]{latency_time}
        \subcaption{Time series}
        \label{fig:latency_time}
	\end{subfigure}
	\begin{subfigure}{0.49\columnwidth}
		\centering
    	\includegraphics[width=1.00\columnwidth]{latency_cdf}
    	\subcaption{Cumulative Density Function (CDF)}
        \label{fig:latency_cdf}
	\end{subfigure}
	\begin{subfigure}{0.49\columnwidth}
		\centering
    	\includegraphics[width=1.00\columnwidth]{latency_pdf}
    	\subcaption{Probability Density Function (PDF)}
    	\label{fig:latency_pdf}
	\end{subfigure}
	\caption{
	    End-to-end latency in the network recorded over the entire 90~min experiments.
	    The longer the PHY range, the lower the latency.
	}
	\label{fig:latency_all}
\end{figure}

\thomas{stopped here}

%------------------------------------------------------------------------------
\subsection{Battery Lifetime}
\label{sec:res_lifetime}

We have the motes report what portion of the time their radio is active (``radio duty cycle''), and what portion of the time the radio is transmitting (``transmit duty cycle'').
We then combine that with the current draw of the radio in transmit and receive states, the supply voltage and the energy contained in a battery to compute a battery lifetime.
This computation does not take into account the possible current draw of other electronics, and assumes the battery is a perfect bucket of charge (i.e, ideal battery).
Even though this is not an accurate prediction of the lifetime of the mote on a real battery, it is good enough to compare the effect on lifetime of the different PHYs \footnote{We are aware that current peaks shorten the effective mote lifetime compared to the ``ideal battery lifetime''.
It turns out that the PHYs leading to the shorter battery lifetime also have the higher peak current.}.


% duty-cycle

We compute a mote's radio duty cycle as $T_{on}/T_{total}$ (see Section~\ref{sec:methodology}).
Fig.~\ref{fig:dutyCycle_time} shows the evolution of the duty cycle over time.
We expect the duty cycle to decrease with data rate.
This does \textit{not} hold for \ofdm\ because of the time to issue SPI commands to the AT86RF215, which can take up to 1~ms as explained in section~\ref{sec:openwsn} and shown in Fig.~\ref{fig:timeslot_templates}.

\begin{figure}
	\centering
	\includegraphics[width=\figwidth\columnwidth]{dutyCycle_time}
	\caption{
	    Evolution of the mote's radio duty cycle over time.
	}
    \label{fig:dutyCycle_time}
\end{figure}

Given $T_{TX}$ and $T_{on}$, we use~\eqref{eq:dc} to compute reception time $T_{RX}$, the transmit duty cycle $DC_{TX}$ and the receive duty cycle $DC_{RX}$.

\begin{equation}
    \left\{
        \begin{array}{lcl}
            T_{RX}  & = & T_{on}-T_{TX}            \\
            DC_{TX} & = & \frac{T_{TX}}{T_{total}} \\
            DC_{RX} & = & \frac{T_{RX}}{T_{total}}
        \end{array}
    \right.
    \label{eq:dc}
\end{equation}

Table~\ref{tab:energy_table} details the calculation of battery lifetime, assuming a mote is powered by a pair of AA batteries holding 8.2~Wh of energy.
\oqpsk\ exhibits a battery lifetime
      5~times larger than \ofdm\ and
     10~times larger than \fsk.
Despite the advantages of \fsk\ and \ofdm\ in range, reliability and latency, their utilization leads to more frequent battery replacement.

\begin{table}
    \centering
    \begin{tabular}{|r|r|r|r|}
    \hline
                          &      \fsk &     \ofdm &     \oqpsk \\ \hline
        DC                &   2.100\% &   0.750\% &    0.650\% \\ \hline
        $DC_{TX}$         &   0.250\% &   0.038\% &    0.050\% \\ \hline
        $DC_{RX}$         &   1.850\% &   0.713\% &    0.600\% \\ \hline
        $I_{TX}$          &     62~mA &     62~mA &      24~mA \\ \hline
        $I_{RX}$          &     28~mA &     28~mA &      20~mA \\ \hline
        Supply voltage    &     2.5~V &     2.5~V &      3.0~V \\ \hline
        energy per day    & 0.0141~Wh & 0.0033~Wh &  0.0015~Wh \\ \hline
        battery lifetime  & 1.6~years & 6.9~years & 15.1~years \\ \hline
        \end{tabular}
    \caption{``Ideal battery'' lifetime for each configuration}
    \label{tab:energy_table}
\end{table}

%==============================================================================
\section{Conclusion}
\label{sec:conclusion}

% summary

Table~\ref{tab:summary} summarizes the key performance indicators of our network measured in the experimental campaign.
\fsk\ exhibits the best network formation time, end-to-end reliability and end-to-end latency,
    at the cost of a battery lifetime roughly 10~times lower than \oqpsk.
\ofdm\ shows balanced results, between \fsk\ and \oqpsk.
There is no single PHY layer that exhibits the best performance over all KPIs.

\begin{table}
    \centering
    \begin{tabular}{|r|r|r|r|r|}
        \hline
                & network formation & end-to-end reliability & end-to-end latency & battery lifetime    \\
            & \multicolumn{1}{c|}{(Section~\ref{sec:res_formation})}
            & \multicolumn{1}{c|}{(Section~\ref{sec:res_reliability})}
            & \multicolumn{1}{c|}{(Section~\ref{sec:res_latency})}
            & \multicolumn{1}{c|}{(Section~\ref{sec:res_lifetime})} \\ \hline
        \fsk    &    \textbf{7~min} &      \textbf{100.00\%} &      \textbf{10~s} &           1.6~years \\ \hline
        \ofdm   &             9~min &                99.84\% &               25~s &           6.9~years \\ \hline
        \oqpsk  &            11~min &                98.08\% &               35~s & \textbf{15.1~years} \\ \hline
    \end{tabular}
    \caption{
        Summary of the Key Performance Indicator measured in our testing.
        The best values are shown in \textbf{bold}.
    }
    \label{tab:summary}
\end{table}

% why PHY should I use?

\textit{So which PHY should I use?}
If choosing a single PHY, which to pick depends on the application.
If it is acceptable to change batteries every year or so, \fsk\ appears to be the most appropriate.
If battery lifetime is of utmost importance, Table~\ref{tab:summary} suggests \oqpsk\ is the most appropriate.
Finally, for an ``in between'' performance, choose \ofdm.  

% flawed

The problem is that the reasoning in the paragraph above is flawed in at least two ways.
First, the results presented in this article, while conducted and presented in a rigorous fashion, only hold for use cases very similar to the deployment shown in Fig.~\ref{fig:building_motes}.
There are undoubtedly cases (deeply sparse network, heavily unbalanced deployed, \dots) where our results don't hold, and where the ``ranking'' of the different PHYs is different.
Second, even if every precaution is taken to pick the right PHY during the design phase of the network, it is entirely possible that conditions or requirements change during the operation of the network.
This could lead to operating with the ``wrong'' PHY layer, leading to sub-optimal performance.

% dynamic

The real outcome of this paper is not the absolute numbers presented in Table~\ref{tab:summary}.
What that table does indicate is that no PHY is the best for all metrics,
    and that best performance is achieved when \textit{combining} the PHY layers, rather than \textit{picking} one.
This result holds when contemplating additional PHY layer,
    one obvious candidate being LoRa~\cite{adelantado17understanding}.
As alluded to in Section~\ref{sec:introduction}, we are at an exciting stage where we have both radio chips which are able to switch PHY on a frame-by-frame basis, and we have the scheduling technology to orchestrate this PHY-layer agility.
We argue for technology-agile network, in which each mote keeps track of the quality of its link to its neighbors \textit{for each PHY layer}, and uses the PHY layer most appropriate for each frame.
This means a mote may use a different PHY to communicate with different neighbors or to send frames belonging to different classes of traffic. 
This also means that, in a multi-hop scenario, a packet can travel from source to destination using a different PHY at each hop.
In that context, energy efficient neighbor discovery and network consistency are the main challenges, elements we are working on in our current research.

%==============================================================================

\reftitle{References}
\bibliography{rady20free}

\end{document}

%##############################################################################
%##############################################################################
%##############################################################################


%------------------------------------------------------------------------------
\subsection{Network Formation Time}
\label{sec:res_formation}

% definition

The network formation time is measured between the moment the DAG root is selected, and the moment the DAG root has received a data packet from each mote.
The network formation process encompasses
    the time it takes for a mote to synchronize to the network,
    the time it takes for it to complete a security handshake and
    the time it takes for it to acquire a rank.
It is the time a user would have to wait for the network to be fully functional.

% worst case

This is a worst case setup, as we turn on all the motes first, and the gateway last.
Per the 6TiSCH standard, traffic generated by the motes during their secure handshake uses shared cells.
All motes trying to join approximately at the same time will cause a lot of contention of these shared cells, increasing the network formation time.

% results: why is it affected by link budget? 

Fig.~\ref{fig:time_firstpacket_cdf} shows that the \fsk\, \ofdm\ , and \oqpsk\ network is 90\%-formed in 7, 9, and 11~min, respectively.
The higher the link budget (i.e., the longer the range), the faster the network forms.

\begin{figure}
	\centering
	\includegraphics[width=0.45\columnwidth]{time_firstpacket_cdf}
	\caption{Network formation time.}
    \label{fig:time_firstpacket_cdf}
\end{figure}

This because nodes hear more neighbors when using a higher range PHY, so they hear a synchronized node faster.
Also, with a higher communication range, nodes have a higher probability of joining through a neighbor closer to the root; this decreases the number of hops necessary for joining.
This leads the network to reach a steady state with 1) generally higher neighbor reach and 2) lower hops to the root. 
This is verified in the following subsections.



% - - - - - - - - - - - - - -
\subsubsection{Number of Neighbors for a Mote}
\label{sec:results_neighbors_mote}

% what does it mean

This metric is defined as the total number of neighbors registered in the neighbor table at the time the sample is collected.
It is affected by PHY layer link budget because neighbors are registered if a packet has been received with RSSI~$> -80$~dB. 
Therefore, the higher the link budget, the higher the chance that more neighbors are discovered. 

% how is it affected by link budget

The impact of the PHY layer on neighbor reach can be observed in Fig.~\ref{fig:neighbors_all}: the mean and inter-quartile range of discovered neighbors (Fig.~\ref{fig:neighbors_time}) accelerate faster in an inverse correlation with link budget.
By minute 30, an average mote has discovered slightly more than half of the network with \fsk, and still in increase until one hour later.
The range of \ofdm\ overlaps with that of \fsk\ even though it is 16 times faster in bit-rate; also it is better than \oqpsk\ by nearly 35\%  even though it is 3 times faster in bit-rate. 
This is verified in Fig.~\ref{fig:neighbors_cdf} as the CDF of all the data samples in steady state shows the inverse correlation between the number of neighbors and the link budget. 
The PDF in Fig.~\ref{fig:neighbors_pdf}  shows high neighbor reach for \oqpsk\ only in a dense setting: a cluster around 25 neighbors in room~B and a cluster around 14 neighbors in nearby rooms

% conclusion pros and cons

Therefore, on the one hand, a high link budget PHY-layer leads to faster network formation yet with a possible risk of neighbor table overflow in case of dense deployments.
This risk is significant because there is no standardized mechanism to accept any new neighbor (even if it is a ``better'' one) in case neighbor table is saturated -- as outlined in the IETF RFC~\cite{rfc8505}. 
On the other hand, a low link budget PHY-layer may lead to slower network formation, yet it is useful in dense deployments where motes need to discover only nearby neighbors. 

\begin{figure}
	\centering
	\begin{subfigure}{0.49\columnwidth}
    	\centering
    	\includegraphics[width=1.00\columnwidth]{neighbors_time}
        \subcaption{Mean and inter-quartile range of measurements in the time domain }
        \label{fig:neighbors_time}
	\end{subfigure}
	\begin{subfigure}{0.49\columnwidth}
		\centering
    	\includegraphics[width=1.00\columnwidth]{neighbors_cdf}
    	\subcaption{CDF of  the measurements for the network at steady state }
        \label{fig:neighbors_cdf}
	\end{subfigure}
	\begin{subfigure}{0.49\columnwidth}
		\centering
    	\includegraphics[width=1.00\columnwidth]{neighbors_pdf}
    	\subcaption{PDF of  the measurements for the network at steady state }
    	\label{fig:neighbors_pdf}
	\end{subfigure}
	\caption{
	    Reported number of neighbors in the network.
	    The higher the link budget, the more neighbors are discovered by the 6LoWPAN adaptation layer.
	} 
	\label{fig:neighbors_all}
\end{figure}

% - - - - - - - - - - - - - -
\subsubsection{Number of Neighbors for the DAG root}
\label{sec:results_neighbors_root}

% What does DAG rank mean and Why it means less hops? 

This metric is calculated at the mote at the RPL layer as a function of the expected transmission count (eTx) for the links leading to the root, and it is summed at each hop.
In the current implementation, the rank is calculated as $rank = ((3\times numTx/numAck)-2)-minHopIncrease$, where $minHopIncrease = 256$. 
An important feature of the DAG rank is that it monotonically decreases along any path to the DAG root, as mandated by the RPL standard~\cite{rfc6550}.
Therefore, lower DAG ranks imply overall better path quality, reflecting a combination of lower number of hops and better links. 

% why does it improves by link budget

A higher link budget at the PHY layer leads to an improved DAG rank and shorter paths to the root as it reduces the need for re-transmissions and for relaying; which leads to faster network formation.
It can be observed in Fig.~\ref{fig:dagrank_time} that \fsk\  maintains, generally, the lowest DAG ranks during both the network formation period and the ensuing steady state. 
Link budget seems to be inversely correlated with the distribution of the DAG rank in the network as observed in Fig.~\ref{fig:dagrank_cdf}: \fsk\ demonstrates the highest DAG rank at steady state for 90\% of the motes, followed by \ofdm\ and \oqpsk\ respectively.
This observation is stressed in Fig.~\ref{fig:dagrank_pdf} which shows that the center of the distribution of the \fsk\ is at the absolute lowest rank possible while the distributions of \ofdm\ shifts to higher ranks followed by \oqpsk. 

% conclusion pros and cons 

Therefore, a higher link budget PHY-layer leads to improved DAG ranks and faster network formation, yet it creates a risk of overloading the nodes with good DAG ranks (closer to the root) with scheduling requests.
Since these requests are established on shared cells, they may cause contention leading to energy waste and potentially packet loss -- which is exacerbated in dense deployments.

\begin{figure}
	\centering
	\begin{subfigure}{0.49\columnwidth}
    	\centering
    	\includegraphics[width=1.00\columnwidth]{dagrank_time}
        \subcaption{Mean and inter-quartile range of measurements in the time domain }
        \label{fig:dagrank_time}
	\end{subfigure}
	\begin{subfigure}{0.49\columnwidth}
		\centering
    	\includegraphics[width=1.00\columnwidth]{dagrank_cdf}
    	\subcaption{CDF of  the measurements for the network at steady state }
        \label{fig:dagrank_cdf}
	\end{subfigure}
	\begin{subfigure}{0.49\columnwidth}
		\centering
    	\includegraphics[width=1.00\columnwidth]{dagrank_pdf}
    	\subcaption{PDF of  the measurements for the network at steady state }
    	\label{fig:dagrank_pdf}
	\end{subfigure}
	\caption{
	    Reported DAG Ranks in the network.
	    The higher the link budget at PHY layer, the better the DAG rank and the faster the root can be reached.
	} 
	\label{fig:dagrank_all}
\end{figure}