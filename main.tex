\documentclass[journal]{IEEEtran}

\usepackage[utf8]{inputenc}
\usepackage{graphicx}
\usepackage{verbatim}
\usepackage{color}
\usepackage{xcolor}
\usepackage{xspace}

\graphicspath{{figs/}}

\newcommand{\fsk}          {FSK\_868}
\newcommand{\oqpsk}        {O-QPSK\_2.4}
\newcommand{\ofdm}         {OFDM\_868}

\newcommand{\todo}[1]      {\textbf{\textcolor{red}{[TODO] #1}}}
\newcommand{\lorem}        {\textcolor{green}{Lorem ipsum dolor sit amet, consectetur adipiscing elit, sed do eiusmod tempor incididunt ut labore et dolore magna aliqua. Ut enim ad minim veniam, quis nostrud exercitation ullamco laboris nisi ut aliquip ex ea commodo consequat. Duis aute irure dolor in reprehenderit in voluptate velit esse cillum dolore eu fugiat nulla pariatur. Excepteur sint occaecat cupidatat non proident, sunt in culpa qui officia deserunt mollit anim id est laborum.}}
\newcommand{\mina}[1]      {\textbf{\textcolor{blue}{[Mina] #1}}}
\newcommand{\quentin}[1]   {\textbf{\textcolor{blue}{[Quentin] #1}}}
\newcommand{\dominique}[1] {\textbf{\textcolor{blue}{[Dominique] #1}}}
\newcommand{\thomas}[1]    {\textbf{\textcolor{blue}{[Thomas] #1}}}

\begin{document}

\title{No Free Lunch: 6TiSCH Performance\\on Different Modulations}

\author{
    \IEEEauthorblockN{
        Mina~Rady\IEEEauthorrefmark{1}\IEEEauthorrefmark{2},
        Quentin~Lampin\IEEEauthorrefmark{1},
        Dominique~Barthel\IEEEauthorrefmark{1},
        Thomas~Watteyne\IEEEauthorrefmark{2}
    }\\
    \IEEEauthorblockA{
        \IEEEauthorrefmark{1}~Orange Labs, Meylan, France
    }\\
    \IEEEauthorblockA{
        \IEEEauthorrefmark{2}~EVA team, Inria, Paris, France
    }
    \thanks{Corresponding author: Mina~Rady (email: {\tt poipoi})}https://www.overleaf.com/project/5e99476cb8ddc80001a31015
}

\maketitle

\begin{abstract}
   \lorem
   \lorem
   \lorem
\end{abstract}

\tableofcontents

%==============================================================================
\section{Introduction}
\label{sec:introduction}

% intro low-power wireless

Low-power wireless communications have undergirded the fourth industrial revolution by means of enabling pervasive machine-to-machine communications, otherwise known as the Internet of Things.
The range of applications of such technologies vary from smart-building monitoring [ref], environmental monitoring [ref], precision agriculture [ref], utility meter remote reading [ref], localization of expensive equipment [ref], smart grid monitoring [ref] and industrial equipment monitoring for preemptive maintenance [ref].
Therefore, the financial value provided by such technology pour into a large array of economic sectors.
A plethora of radio technologies have been introduced to serve the IoT communications demands in different contexts and with different features.
For example, LoRa, a family of LoRa modulations, introduce a low-bitrate high link-budget modulations for long range communications.
LoRaWAN architectures are star-based topologies and they operate and Aloha-based Medium Access Control, which non-deterministic based on the level of contention in the network [ref limitations of lorawan]

% popular PHY layers
% \lorem IEEE 802.15.4
Another major player in the IoT communications market is the set of technologies under the IEEE 802.15.4 family of standards which were conceived in 2003 yet extended in several amendments since 2011. [ref]
The first amendment is the IEEE 802.15.4e standard which defines a MAC sublayer that follows the Time-Slotted Channel Hopping (TSCH) paradigm for low-rate wireless personal area networks. [ref]
In contrast to a non-deterministic protocol as LoRaWAN, the TSCH paradigm is capable of providing wire-like reliability due to the strict allocation of time and frequency resources in the network based on a hashing scheme of node IPv6 addresses. [ref]
By means of this allocation, transmissions occur, mostly, in dedicated combinations of time-slots and channels (also known as cells).
This allows the mitigation of the side-effects of contention, although it does not completely remove them because a minor subset of network management transactions have to take place still in "shared" cells, where contention is possible.

The third amendment is the IEEE 802.15.4g standard which which defines physical layer specifications for low-data-Rate, wireless, smart metering utility networks. [ref]
This specification outlines three families of multi-rate and multi-regional (MR) modulations: 1) Frequency Shift Keying (MR-FSK), 2) Offset quadrature phase-shift keying (MR-O-QPSK), and 3) Orthogonal frequency division multiplexing (MR-OFDM).
Furthermore, several frequency bands are defined as part of the standards as common communication "channels" in the license-free industrial, scientific and medical (ISM) band.
While the ISM frequency bands may vary depending national regulations, they are usually split under two categories: the 2.4 GHz band and the sub-GHz band. 
The 2.4 GHz band is usually characterized by lower link-budget and shorter range than the sub-GHz band (due to the higher frequency). 
This makes it sufficient for small home-area networks, yet it makes it highly susceptible to interference from common technologies in the same band such as WiFi [ref]
The sub-GHz band is usually characterized by higher link budget and longer range than the 2.4 GHz band. 
This makes it suitable for challenging deployments such as kilometer-scale links in rural areas or complex indoor environments. [ref munoz]
The modulations in this amendment support a wide range of bitrates as low as 25 kbps and as high as 800 kbps (in the European sub-GHz band of 868 MHz)

% capabilites of recent chips

% \lorem nRF52840, CC2538, CC2650
Several radio chips and System-on-Chip (SoC) products have been increasingly introduced to the market in compliance the IEEE 802.15.4g standard. 
As the technology advanced, chips began to offer multiple modulations and even multiple frequency bands depending on how their registered are configured by program running on the micro-processor.
Common examples of these systems are CC2538 [ref] and CC2650 [ref] SoCs by Texas Instruments and nRF52840 [ref] by Nordic Semiconductor. 
They are built around a 32-but ARM cortex with electrical features that are dedicated for low-power performance.
Their radio-chip design allows the choice of both modulation and frequency band by the executing program. 
Furthermore, depending if the host board has multiple antennas as part of its electronic design, the executing program maybe be able to choose which antenna to use based on the choice of frequency band. 
The example host board used for this paper is the OpenMoteB which features a 2.4GHz antenna and a sub-GHz antenna. 
The board design allows dynamic switching between antennas and the radio chips on board at run-time. [ref]

% goal
The IPv6 Time Slotted Channel Hopping (6TiSCH) protocol stack has been extensively standardized by the Internet Engineering Task Force (IETF). 
The MAC layer of 6TiSCH is the IEEE 802.15.4e standard allowing industrial-grade reliability.
The PHY layer of 6TiSCH is the IEEE802.15.4g standard allowing a wide variety of selection among the aforementioned modulations supported by the standard.
Current refrence architectures of 6TiSCH implementations relied almost canonically on O-QPSK PHY layer in the 2.4 GHz band.[ref]
On one hand, this limits the stack performance by the limitations of the short range \oqpsk radio and discards potential advantages of other modulations such as higher link-budget, leading to km-scale connectivity or very high bit-rate up to 800 kbps.
On the other hand, there is no evidence, to the best of our knowledge, on how the change of the PHY layer could impact the network performance as a whole. 

Therefore, he goal of this paper is to compare the system-level performance of the 6TiSCH stack on top three PHY layers of complementary characteristics: 
     1)A high-frequency, high bit-rate option: O-QPSK in the 2.4 GHz band, offering a nominal bitrate of 250 kbps.   
     2) A low bit-rate, low frequency option: FSK option 1 in the 868 MHz band, offering nominal bit-rate of 50 kbps. 
     3) A very high bit-rate, low frequency option: OFDM option 1  in the 868 MHz band, with Modulation and Cosing Scheme (MCS) 3, offering a nominal bit-rate of 800 kbps, and 

 
The contributions of this article are three-fold:

\begin{itemize}
   \item Using \fsk the most number of neighbors are discovered. It leads to the fastest network formation time with the lowest latency and lowest network depth.Yet it kills network battery lifetime. It risks neighbor table overflow in high density deployments.
   \item Using \oqpsk, the least number of neighbors are discovered. It leads to the slowest network formation time, highest RPL depth (lowest DAG ranks) and effectively the highest end-to-end latency. It consumes the most packet buffer on average because of storage needed to re-transmissions, leading to a higher risk of packet-buffer overflow. 
   \item Using \ofdm, number of discovered neighbors is in between \fsk and \oqpsk, despite being the highest bit-rate in the three modulations. It shows an intermediate neighbor discovery range between both modulations. It shows surprisingly a network formation speed, and end-to-end latency comparable to \fsk. Yet it has a duty cycle comparable to OQPSK. It shows the least steady-state PDR among the three.
\end{itemize}
  


%==============================================================================
\section{Related Work}
\label{sec:related_work}

% 154g range testing, Munoz: Gap: long range could lead to deteriorating side effects on end to end performance such as neighbor table overflow and neighbor table overflow

% Multiband support Martina: We show that 

% Novelty compared to state of the art:
% Full end to end testing before

% contributions
\mina {Updated Findings in a nutshell (for abstract?):}
\begin{itemize}

\item  \fsk is best fo fast network formation, least end-to-end latency, and least packet buffer memory footprint, but risks neighbor table overflow in dense deployments.

\item \oqpsk is worst for latency but best for neighbor discovery in dense deployments and for battery life time.
 
\item  \ofdm, even though it is the highest bitrate, it shows a comporomise between both radios. It falls in between both in terms of neighbor discovery range, battery lifetime, memory footprint, end-to-end latency. Despite its long range comparable to FSK, it has an overall low duty cycle comparable to OQPSK. And despite its high bitrate, it shows range capability comparable to FSK.
\end{itemize}



% remainder

The remainder of this article is organized as follows.
Section~\ref{sec:opentested} \todo{}.

%==============================================================================
\section{The OpenTestbed}
\label{sec:opentested}

% intro testbed, architecture

\lorem

% OpenMote and testbox

\lorem

\begin{figure}
	\centering
	\includegraphics[width=0.90\columnwidth]{mote_ot}
	\caption{OpenMote B board (left) and The testbox containing four OpenMote B boards (right).}
    \label{fig:testbox}
\end{figure}

% 42-mote deployment

\lorem

\begin{figure}
	\centering
	\includegraphics[width=0.90\columnwidth]{building_motes}
	\caption{Floorplan of the deployment.}
    \label{fig:floorplan}
\end{figure}

%==============================================================================
\section{A PHY-layer Agile Extension of OpenWSN}
\label{sec:openwsn}

% intro to 6TiSCH

\lorem

% intro to OpenWSN

\lorem

% goal: multiple PHY layer

\lorem \fsk \oqpsk \ofdm``FSK\_subGHz''

% OpenWSN Extension

\lorem

% result: footprint

\lorem

%==============================================================================
\section{Methodology}
\label{sec:methodology}

% repeating 3 times

\lorem

% duration of one experiment

\lorem

% gathering data, publishing raw results, etc.

\lorem

% KPIs

\lorem

% running the experiments

\lorem

%==============================================================================
\section{Experimental Results}
\label{sec:results}

\lorem

%------------------------------------------------------------------------------
\subsection{Network Formation}
\label{sec:network_formation}

% why is it important, define

\lorem

% worst case from a contention point of view

\lorem

% results

\lorem Fig.~\ref{fig:time_firstpacket_cdf}

\begin{figure}
	\centering
	\includegraphics[width=0.90\columnwidth]{time_firstpacket_cdf}
	\caption{Time to First Packet CDF.}
    \label{fig:time_firstpacket_cdf}
\end{figure}

% settling time, steady-state   

\begin{figure}
	\centering
	\includegraphics[width=0.90\columnwidth]{aggregate_plot_reliability}
	\caption{6TiSCH reliability performance under each radio setting in the three stages of network formation: 1) discovery and formation phase (red highlight),and 2) steady-state phase (blue highlight). Shaded curves represent the majority of the distribution (interquartile range)}
    \label{fig:aggregate_plot_reliability}
\end{figure}


%------------------------------------------------------------------------------
\subsection{End-to-End Reliability}
\label{sec:reliability}
% why is it important, and how is it measured? 
% Packet loss on shared cells. 
% Long range lead to more neighbors which leads to discovery of better neighbors closer to the root. 

% The negative side is that requires more active and strict rule for accepting neighbors, otherwise it would lead to a neighbor table buffer overflow, specially in dense networks

% Short range can cause more packet loss due to lower link budget if propagation is challenging. 


\begin{table}
 \caption {Steady-state end-to-end PDR values} \label{tab:pdr_table} 
 \begin{center}
 \begin{tabular}{||c c c c||} 
 \hline
 Radio Setting & Average & Median & Stdev \\ [0.5ex] 
 \hline\hline
 \fsk & 100\% & 100\% & 0 \\ 
 \hline
 \ofdm & 99,9521\% & 100\% & 0.302 \\
 \hline
 \oqpsk & 100\% & 100\% & 0 \\
 \hline

\end{tabular}
\end{center}
\end{table}


\begin{figure}
	\centering
	\includegraphics[width=0.9\columnwidth]{avg_pdr_plot.png}
	\caption{Average end-to-end PDR. }
    \label{fig:avg_pdr_plot}
\end{figure} 

%------------------------------------------------------------------------------
\subsection{End-to-End Latency}
\label{sec:latency}

% why is this important? information recency, alarms, logging of critical events for the reactive or preemptive measures.
% higher link budget lead to discovery of more neighbors which leads less hops and less end to end latency.

% shorter range not only suffer from more hops but also more re-transmissions.

\begin{figure}
	\centering
	\includegraphics[width=0.90\columnwidth]{avg_latency_plot}
	\caption{Average end-to-end latency. Shaded curves represent the majority of the distribution - i.e. interquartile range (extracted from Fig. \ref{fig:aggregate_plot_reliability})}
    \label{fig:avg_latency_plot}
\end{figure}

%------------------------------------------------------------------------------
\subsection{Queue Occupancy}
\label{sec:queue}


% why is this important? memory limitations, packet loss simply due to unavailable buffer space. 

% Interestingly enough, it seems that there is an inverse correlation between bit-rate and average packet buffer occupancy as seen in Fig.~\ref{fig:avg_bufferSize_plot}. 

% In this setting, the packet queue buffer is of size 20 entries. 



\begin{figure}
	\centering
	\includegraphics[width=0.90\columnwidth]{avg_bufferSize_plot}
	\caption{Average queue occupancy (extracted from Fig. \ref{fig:aggregate_plot_reliability})}
    \label{fig:avg_bufferSize_plot}
\end{figure}

%------------------------------------------------------------------------------
\subsection{Battery Lifetime}
\label{sec:battery_lifetime}


% why is this important? essential opex. OpEx presents a huge constraint and could lead to impractical deployment if batteries have to change too often (thesis reference)
\begin{figure}
	\centering
	\includegraphics[width=0.90\columnwidth]{avg_avg_dutyCycle_plot}
	\caption{Average duty cycle.Shaded curves represent the majority of the distribution - i.e. interquartile range. (extracted from Fig. \ref{fig:aggregate_plot_reliability})}
    \label{fig:avg_avg_dutyCycle_plot}
\end{figure}

%------------------------------------------------------------------------------
\subsection{Cumulative Comparison of the Results}
\label{sec:cimulative}



\begin{figure}
	\centering
	\includegraphics[width=0.90\columnwidth]{latency_cdf_plot_full}
	\caption{CDF of latency in the network} 
    \label{fig:latency_cdf_plot_full}
\end{figure}

\begin{figure}
	\centering
	\includegraphics[width=0.90\columnwidth]{dutyCycle_cdf_plot_full}
	\caption{CDF of radio duty cycle in the network} 
    \label{fig:dutyCycle_cdf_plot_full}
\end{figure}

\begin{figure}
	\centering
	\includegraphics[width=0.90\columnwidth]{numNeighbors_cdf_plot_full}
	\caption{CDF of number of neighbors in the network} 
    \label{fig:numNeighbors_cdf_plot_full}
\end{figure}

\begin{figure}
	\centering
	\includegraphics[width=0.90\columnwidth]{maxBufferSize_cdf_plot_full}
	\caption{CDF of max packet buffer occupancy in the network} 
    \label{fig:maxBufferSize_cdf_plot_full}
\end{figure}
\begin{figure}
	\centering
	\includegraphics[width=0.90\columnwidth]{dagRank_cdf_plot_full}
	\caption{CDF of DAG Rank in the network} 
    \label{fig:{dagRank_cdf_plot_full}}
\end{figure}




\mina{mention the exact tx power levels you are using}
\begin{table*}[t]
 \caption {Battery Life - assuming 2 AA batteries with total 3V and 4.2 wh capacity} \label{tab:energy_table} 
 \begin{center}
\begin{tabular}{lllllllll}
      & Total DC & Tx DC (measured) &  \begin{tabular}[c]{@{}l@{}}RX DC\\ (estimated)\end{tabular} & \begin{tabular}[c]{@{}l@{}}Tx Current \\ (mA)\end{tabular} & \begin{tabular}[c]{@{}l@{}}Rx Current\\  (mA)\end{tabular} & \begin{tabular}[c]{@{}l@{}}Tx Power Consuption \\ (wh)\end{tabular} & \begin{tabular}[c]{@{}l@{}}Rx Power Consumption\\  (wh)\end{tabular} & \begin{tabular}[c]{@{}l@{}}Battery lifetime\\  (days)\end{tabular} \\
\fsk   & 0,02     & 0,00225          & 0,01775                                                     & 62                                                         & 28                                                         & 0,186                                                               & 0,084                                                                & 251,4997                                                           \\
\ofdm  & 0,00875  & 0,000375         & 0,008375                                                    & 62                                                         & 28                                                         & 0,186                                                               & 0,084                                                                & 872,3785                                                           \\
\oqpsk & 0,0077   & 0,0005           & 0,0072                                                      & 24                                                         & 20                                                         & 0,072                                                               & 0,06                                                                 & 2607,892                                                          
\end{tabular}
\end{center}
\end{table*}
%------------------------------------------------------------------------------
\section{Discussion}
\label{sec:discussion}
\begin{table}[]
 \caption {6TiSCH performance ranking for each setting} \label{tab:summary}
\begin{tabular}{lllllllll}
      & network fomration speed. & energy & latency & buffer effeciency & pdr &  &  &  \\
\fsk   & 1                        & 3      & 1       & 1                 & 1   &  &  &  \\
\ofdm  & 1                        & 2      & 2       & 1                 & 2   &  &  &  \\
\oqpsk & 3                        & 1      & 3       & 2                 & 1   &  &  & 
\end{tabular}
\end{table}


%\begin{figure}
%	\centering
%	\includegraphics[width=0.90\columnwidth]{avg_avg_dutyCycleTx_plot}
%	\caption{Average transmission duty cycle}
%   \label{fig:avg_avg_dutyCycleTx_plot}
%\end{figure}

%------------------------------------------------------------------------------


%==============================================================================


\end{document}

%==============================================================================
\section{Summary}

The experiments evaluate the impact of utilization of different radio access technologies at the physical layer of the 6TiSCH stack.
\begin{itemize}
    \item FSK 1 in the subghz band
    \item OFDM 1 MCS3 in the subghz band
    \item OQPSK in the 2.4 GHz band
\end{itemize}

The stack performance is evaluated at a end-to-end looking at  KPIs relevant to an SLA for a critical infrastructure, namely:
\begin{itemize}
    \item Network formation time: defined as CDF of time to first packet from all connected motes within a 30 minute time span.
    \item Reliability: defined in terms of PDR
    \item Quality of Service: defined as traffic latency
    \item Battery lifetime: expected lifetime with a power supply of 2 AA batteries.
    \item Resilience: defined as combination of two metrics: PDR of connected motes and ratio of disconnected motes (i.e. motes that were once connected) after sudden total failure of 10\% of the motes in the network (fixed set). 
\end{itemize}
    
Furthermore, the stack performance on top of each radio is evaluated at system-level for the following scenarios:
\begin{itemize}
    \item Increased traffic demand (by 300\%)
    \item Lower Network size and density (by 50\%)
\end{itemize}

%==============================================================================
\section{Draft outline}

\begin{enumerate}
    \item We explore the performance of the 6TiSCH stack in heterogeneous radio settings and we comment on the following aspects of the stack performance: network formation, reliability, quality of service, and power consumption. We observe the following. \item FSK leads to faster network formation (section \ref{sec:network_formation}). 
    \item However, FSK , as robust as it is, can lead to overall lower end-to-end reliability , which is contrary to the case of OFDM (section \ref{sec:reliability}) 
    \item Despite the lower PDR of the 6TiSCH stack in top of FSK, it shows much more end-to-end stable latency. (section \ref{sec:quality_of_service})
    \item Shorter range radios , even though they consume less energy in principle, they have a side effect as they can lead to the occupancy of the packet memory buffer due to consistent re-transmissions. This could risk reaching buffer overflow as network traffic increases. (section \ref{sec:quality_of_service})
    \item Furthermore, higher reliability of OFDM does not mean higher resilience. 
    Even though OFDM shows 99.6-100\% PDR on average compared to 99.0-99.5\% average PDR for FSK, FSK shows the best resilience in terms of PDR maintenance.  
    FSK result into a higher network degree which risks reaching limits of allocated memory buffers allocated for storage of neighbor information (section \ref{sec:resilience})
    \item Despite the advantages of FSK for end-to-end latency and network formation speed, its power consumption present a significant drawback. Also, despite the higher bitrate of OFDM compared to OQPSK, it ends up consuming similar duty cycles (b/c of re-transmissions?) (section \ref{sec:battery_lifetime})
\end{enumerate}

\begin{figure}
	\centering
	\includegraphics[width=0.90\columnwidth]{avg_avg_dutyCycleTx_plot}
	\caption{Average duty cycle. Shaded curves represent the majority of the distribution (interquartile range).}
    \label{fig:avg_avg_dutyCycleTx_plot}
\end{figure}